\chapter{Visualising networks on the Web}

Even though the data mining processes described in the previous chapters give us valuable insights into the structure of social networks, 
they are not necessarily easy to interpret for laymen.
One way to make the results of data mining more accessible is to create data visualization, that is present the data with their visual representation, 
while using different visual cues to guide the viewers' attention towards different qualities.

In this part of the thesis, we will assess the current state of the visualizations in the Charles Explorer application, 
will propose some improvements to make the visualization more accessible to the users and will implement those.

At the moment of publishing this thesis, the person search mode in the Charles Explorer application\footnote{Available e.g. at \url{https://explorer.cuni.cz/person/1732969562160398}} already contains the experimental prototype of the changes proposed in this chapter.
For consistency, we will still consider the ``current state'' of the visualization the one \textit{without} the proposed changes.

The implementation of the visualization components is available in the GitLab repository\footnote{At \url{https://gitlab.mff.cuni.cz/barj/charles-explorer/}}.

\section{Vocabulary}

This section introduces some basic vocabulary that we will use throughout this chapter.

\textbf{Visual decoding}: Also called \textit{preattentive processing} or \textit{preattentive vision}, visual decoding is the instantaneous perception of the visual field that comes without apparent mental effort. (\cite{Cleveland1985})

\textbf{Bipartite network projection}: A bipartite network is a graph whose nodes can be divided into two disjoint sets, $U$ and $V$, such that every edge connects a node in $U$ to a node in $V$.
A (monopartite) projection of a bipartite network is a graph that represents the connections between the nodes in one of the sets, $U$ or $V$.
The nodes of the set are connected in the projection if they share a common neighbor in the bipartite network.

\section{Assessing the current state} \label{sec:current-state}

The current state of the Charles Explorer visualization views is quite simple. 

In the \textit{Person} search mode, the user can search for people inside the Charles University. 
When accessing a person's profile, the application shows the person's \textit{ego network} with the main person and their direct collaborators 
as nodes and their common publications aggregated to the edges.

\begin{figure}[ht!]
    \captionsetup{width=.9\linewidth}
    \includegraphics[width=0.8\textwidth]{../img/charles-explorer-old-view.png}
    \centering
    \caption{Charles Explorer showing the \textit{ego network} of a person.}
\end{figure}

The graph is displayed with force-directed layout. The edge thickness is proportional to the number of common publications between 
the two people and the colors of the nodes represent the person's faculty association.

This approach has multiple drawbacks which we will now discuss.

\subsection{Problems with color coding} \label{sec:color-coding}

Firstly, the color coding of the nodes does not prove useful, as it hinders the \textit{visual decoding} of the graph.
The user spends attention on reading the legend, rather than interpreting the graph subconciously.

This is especially true for larger ego networks with many nodes with different faculty affiliations. 
Additionaly, the application does not provide any alternative visual cue for color vision deficient users.

\begin{figure}[ht!]
    \captionsetup{width=.9\linewidth}
    \includegraphics[width=0.7\textwidth]{../img/color-coding.png}
    \centering
    \caption{Graph view for query \textit{`dentistry'} shows nodes with various faculty affiliations.}
\end{figure}

According to \cite{Cleveland1985}, the upper bound on color discrimination in one figure is 5–6 colors for a healthy viewer. 
This is not enough for the 17 faculties and departments of the Charles University. 
With faculties, there is also little room for a meaningful aggregation (of more faculties into one color), as the faculty structure is not hierarchical.

\subsection{Layouting problems} \label{sec:layouting-problems}

The arbitrary positions of the nodes based on the physical simulation layout increase the cognitive load 
of the viewer and contribute to the graph's worse readability.

\cite{munzner2015visualization} states that the position of data points on a common scale is the most effective way to communicate the data to the viewer.
By using the physical simulation layout, we are willingly giving up this visual channel (the \texttt{x} and \texttt{y} position of the nodes in the screen space).

In case the user is looking for a specific person in the graph, they have to scan the whole graph to find the person, as the node position does not code any inherent information about the person.

\subsection{Contracting publication nodes into edges}

The current data visualization uses authors as the nodes and contracts publications as the edges.
The number of common publications is aggregated to the edge thickness. \cite{10.5555/2385879} considers thickness a visual channel with a limited quantitative resolution.

Setting the edge width in proportion to the number of common publications might also cause confusion, as the \textit{area} of the edge depends both on the width and the length of the edge.
Longer edges might appear more important than the shorter ones, even though they might denote the same number of common publications.

This is undesirable, as it goes against the underlying idea of the physical simulation layout, which places the nodes closer to each other if they are more connected.

Furthermore, the contraction of the publication nodes into edges poses another problem.
A publication with $n$ authors contributes to $\frac{n(n-1)}{2}$ edges between its authors. 

\begin{figure}[ht!]
    \captionsetup{width=.9\linewidth}
    \includegraphics[width=0.7\textwidth]{../img/contraction.png}
    \centering
    \caption{Contracting publication nodes into edges only shows coauthorship between pairs of authors. Larger communities of coauthors (\# coauthors > 2) are not displayed correctly in the graph.}
\end{figure}

This means that publications with one author are not represented in the graph at all, as there are no edges to draw between the nodes.

For publications with more than two authors, one publication is represented by multiple edges between all the pairs of the authors.
This clutters the graph with edges and makes the publication - edge mapping less readable. 
For correct representation of those publications, we would need to draw hyperedges between the nodes, which might cause the readability of the graph to decrease even further.

\section{Addressing the issues}

To address the problems from the section \ref{sec:current-state} and improve the visualization of the ego networks, we propose changes to the current visualization.
We also implement an experimental prototype visualization of the proposed changes.

\subsection{Ego-network visualization}

To start, we can address the problem of the \textit{publication-edge contraction} by displaying the publications as nodes in the graph.

This way, both the entity types (authors and publications) are displayed as \textit{nodes} in the visualization.
This is perhaps easier to understand for a layman, as both authors and publications are real-life entities. 
The edges between the nodes now represent an incidence relation between the two entities.

The resulting graph is a bipartite graph, with two types of nodes - authors and publications, and edges connecting the authors to the publications they have co-authored.

To support the \textit{visual decoding} of the graph, we distinguish the two types of nodes by their shape. 
This way, the user can easily distinguish between the authors and the publications, even if they are color vision deficient 
or are viewing the visualization reproduced using a monochrome display medium (for example a print from a black-and-white printer).

The main node (the ego) is highlighted with color fill - since it is the only node of this type in the graph,
it is easy to distinguish from the other nodes - even with monochrome display mediums or color vision impairments.

\begin{figure}[ht!]
    \captionsetup{width=.9\linewidth}
    \includegraphics[width=0.7\textwidth]{../img/publications-and-people.png}
    \centering
    \caption{In the proposed visualization, the publications are displayed as nodes in the graph. \textit{Larger-than-binary} coauthorships are now represented correctly.}
\end{figure}

This approach also removes the need for the edge-width encoding of the number of common publications.

To avoid the color coding problems from \ref{sec:color-coding}, we defer the faculty affiliation information to the node tooltip.
This is only visible after the user hovers over the node with the mouse cursor, so it does not clutter the graph view.

\subsection{Node locality and layouting}

As mentioned in \ref{sec:layouting-problems}, the arbitrary positions of the nodes in the force-directed layout increase the cognitive load of the viewer.
For larger graphs, the viewer might not be able to find the node they are looking for, as the node position does not code any inherent information about the person.

To address this, we propose a search tool which highlights the search results in the graph.

\begin{figure}[ht!]
    \captionsetup{width=.9\linewidth}
    \includegraphics[width=0.8\textwidth]{../img/big-network.png}
    \includegraphics[width=0.8\textwidth]{../img/big-network-search.png}
    \centering
    \caption{The first picture shows a large ego network with many nodes. The second picture shows the same network with entities highlighted by the search tool (search for \textit{video} shows related publications and co-authors).}
\end{figure}

To further improve the graph readability and the usability of the tool, we propose a \textit{query suggester}. 
This is an user interface element consisting of multiple buttons with predefined queries, which the user can click to search for the query in the graph.
The suggested queries are based on tf-idf analysis of the text data connected to the ego network (publication titles and abstracts). 

The tokens (unigrams and bigrams) with the highest tf-idf score are selected as the suggested queries.

\subsection{Faculty affiliation}

While we have already addressed the color coding issues from \ref{sec:color-coding} by introducing the on-hover tooltip, 
this has removed some of the information from the graph view.

In the original visualization, the faculty affiliation for people was displayed as the node color.
This has helped the user to quickly identify the faculty affiliation of the ego's collaborators, 
but also to see the distribution of the  faculty affiliations in the ego network, 
and identify interesting collaboration patterns. 
This is no longer possible with the new tooltip-based approach.

To address this, we propose a new visualization of the faculty affiliation data.
Each co-author of the ego has a faculty affiliation assigned to them, along with the number of common publications with the ego.

For visualizing the distribution of faculty affiliations in the ego network, we can use a \textit{pie chart}.

\begin{figure}[ht!]
    \captionsetup{width=.9\linewidth}
    \includegraphics[width=0.8\textwidth]{../img/pie-chart.png}
    \centering
    \caption{The \textit{pie chart} view shows the distribution of faculty affiliations in the ego network, as well as the most frequent collaborators of the ego.}
\end{figure}

This visualization is still suffering from some of the problems of the original one. 

A publication with $n$ authors still contributes to $(n-1)$ pie chart arcs. 
This causes the solo publications to not be represented in the pie chart at all, and the publications with more than two authors to be overrepresented.

The color coding of the pie chart arcs (denoting the faculty affiliation) is also not ideal, since it still poses a problem for color vision deficient users.

For addressing these issues, we add few more features to the pie chart view.

Firstly, we reuse the tooltip user interface element from the graph view. 
When the user hovers over the pie chart arc, they can see the faculty affiliation name and the number of co-authored publications.

This helps both the color vision deficient users with identifying the faculty affiliation, and the general users with understanding the scale of the collaboration.

Secondly, we add an on-click boolean \texttt{AND} filter. 
When the user clicks on the pie chart arc, the view filters the underlying data to only represent co-authors of both the ego and the clicked-on co-author.

\begin{figure}[ht!]
    \captionsetup{width=.9\linewidth}
    \includegraphics[width=0.8\textwidth]{../img/pie-chart-filter.png}
    \centering
    \caption{Right-clicking on the pie chart arc shows intersection of the ego's and the clicked-on person's collaborators.}
\end{figure}

The \texttt{AND} filter can be used repeatedly, adding new ``pinned'' collaborators one at a time.
Focusing on a smaller subset of the ego network can help to mitigate the problems with the overrepresentation of the publications with more than two authors
help the user to understand the smaller-scale collaboration patterns better.

The filter also applies to the left side of the view, where the user can see the actual publications of the selected co-authors.
This further clears up the multiple co-authorship problem.

