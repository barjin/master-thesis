\chapter{Visualizing large social networks}

Even though the data mining processes described in the previous chapters give us valuable insights into the structure of social networks, they are not necessarily easy to interpret for laymen.
One way to make the results of data mining more accessible is to create data visualization, i.e. present the data with their visual representation, 
while using different visual cues to guide the viewers' attention towards different qualities.

Data visualization is especially useful when dealing with graph-based data, such as social networks, as they have a natural visual representation.

\section{Graph layouting algorithms}

Before we start, let us introduce some basic terminology that will be used throughout this chapter.

A \emph{graph layout} is a way to position the nodes of a graph in a two-dimensional space, such as on a screen or in print. 
This is a necessary step of any graph data visualisation task - without computing the layout, the graph nodes nor edges do not have any intrinsic location assigned to them.

There are several various ways of computing the graph layout, which we'll describe now in no particular order:

\begin{itemize}
    \item \textbf{Hierarchical layout} - for certain types of graphs, it's beneficial to display them in a hierarchical manner. 
    This is usually the case for trees, i.e. connected acyclic graphs - such as family trees or organisation structures.

    Such layout is typically laid out on either horizontal or vertical axis, with the root node at the top or leftmost position, and the children nodes below or to the right of their parent nodes.
    The other direction (i.e. vertical for horizontal-major graphs and vice versa) is used for layouting the incomparable sibling groups.
    
    While this layout might be used on cyclic graphs as well, it's not as common, as it's not as easy to interpret for laymen as the other layouts.
    It might be useful for graphs with some kind of hierarchy or other clear ``score'' quality, such as social networks with a clear leader, or a company structure.

    \item \textbf{Circular layout} - this layout places the nodes on a circle, with the edges connecting them. 
    This layout is useful for visualizing relations between members of a group or community, as it's easy to see which nodes are connected to which other nodes.
    Some work can be done for calculating the optimal placement of nodes - be it minimizing the edge length, or minimizing the number of edge crossings.

    While minimizing the number of edge crossings might help with the graph readability, it might result in a less visually appealing graph, as the nodes might be placed further apart than necessary.
    % TODO: add a reference to the paper about this.

    Trying to minimize the edge length by ordering the nodes on the circle also results in a more readable graph, as this layout promotes grouping closely connected subcommunities together.

    \item \textbf{Grid layout} - this layout places the nodes on a grid, with the edges connecting them.
    While some optimizations, such as minimizing the edge length can be done in this case as well, 
    this type of layout usually results in a less readable graph.

    An equally-spaced grid layout only communicates the relations between nodes by drawing the edges between them, not utilising the position of the nodes.
    It is often the spatial proximity of the nodes that helps viewers understand the relations between the nodes subconciously - and while we still can 
    supply this information by using different visual cues (e.g. coloring of the nodes), it's often not as intuitive as the spatial proximity.

    \item \textbf{Force-directed layout} - this layout is based on the physical simulation of the forces between the nodes and edges of the graph.
    The nodes are treated as charged particles, and the edges as springs - the nodes repel each other, while the edges try to keep the nodes connected.
    
    This results in a layout where the nodes are placed in such a way that the edges are as short as possible, while the nodes are as far apart as possible.
    This is useful for visualizing the overall structure of the graph, as it's easy to see which nodes are connected to which other nodes, and how the graph is connected.

    Using the edges to guide the layouting process results in a more ``intuitively'' readable graph, as the nodes that are connected are placed closer together, while the nodes that are not connected are placed further apart.

    The force-directed layout is especially useful for visualizing large social networks, as it's easy to see which nodes are connected to which other nodes, and how the graph is connected.
    


\end{itemize}

