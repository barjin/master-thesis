\chapter*{Conclusion}
\addcontentsline{toc}{chapter}{Conclusion}

The goal of this thesis was to improve the Charles Explorer web application - both in terms of user experience and the search engine - using the synthesized \textit{academic social network} from the existing relational data.

We have devised a highly performant pipeline for trasforming the relational data into a graph representation of the academic social network.
After the transformation, we explored the network and proposed various ways of utilizing the network metrics for inferring missing data.

While the hierarchical clustering approach yielded better results than the baseline naïve methods, the performance of the network-based methods was not consistent enough to be used in the production environment.
The improved (context-aware) naïve approach, however, showed promising results and is now a part of the Charles Explorer application.

With the synthesized social network in place, we explored the possibility of using the network for re-ranking the search results in the Charles Explorer application.
By benchmarking different re-ranking strategies against existing academic search engines, we have shown that the social network-based re-ranking can improve the search results' relevance.
%% todo fact check

The best-performing re-ranking strategy was the one based on the PageRank algorithm, which outperformed the baseline by a significant margin.
The computational performance of the re-ranking strategies unfortunately did not satisfy the requirements for the usage in the production environment.

Lastly, we have reimplemented the tool for visualising the academic social network in the Charles Explorer application. 
The tool now better adheres to the visualization principles of graph data, it is more user-friendly and performant.
By simplifying certain parts of the visualization, we improve the interpretability of the visualized data.

This visualization tool became a part of the Charles Explorer application and is now available to the users.
