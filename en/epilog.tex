\chapter*{Conclusion}
\addcontentsline{toc}{chapter}{Conclusion}

The goal of this thesis was to improve the \href{https://explorer.cuni.cz/}{Charles Explorer web application} - both in terms of user experience and the search engine - using the synthesized \textit{academic social network} from the existing relational data.

We \hyperref[sec:target-data-model]{have devised} a highly performant pipeline for trasforming the relational data into a graph representation of the academic social network.
After the transformation, we explored the network and proposed various ways of utilizing the network metrics for \hyperref[sec:inferring-missing-identities]{inferring missing data}.

Surprisingly, the baseline naïve methods \hyperref[fig:hierarchical-f1]{dominated} the network-based methods in the task of inferring missing data.
This is likely due to the specific distribution of person names in our testing dataset.
The \hyperref[sec:on-demand-identity-inference]{improved (context-aware) naïve approach}, showed even more promising results and is now a part of the Charles Explorer application.

With the synthesized social network in place, 
we \hyperref[search-ranking-issues]{expressed concerns} about the quality of full-text search based ranking and we explored the possibility of using the network for re-ranking the search results in the Charles Explorer application.

Deeper comparison with search results of the Elsevier Scopus academic search engine revealed that
professional search engines \hyperref[baseline-benchmark]{seemingly} use only full-text search based ranking too.

By retrieving \textit{citation counts} from the Scopus database, we were able to benchmark different re-ranking strategies using the social network metrics against
an objective measure of academic impact.

While these experiments yielded \hyperref[citation-based-ranking]{promising results} - as the search result reranking based on the social network metrics 
did outperform the full-text search based ranking - the computational performance of the re-ranking strategies did not satisfy the requirements for the usage in the production environment.

Lastly, we have \hyperref[sec:current-state]{assessed} the user experience of the Charles Explorer application 
and identified the problematic parts of the existing visualization tool for the academic social network.

We \hyperref[sec:addressing-issues]{reimplemented the tool} for visualising the academic social network in the Charles Explorer application. 
The tool now better adheres to the visualization principles of graph data, it is more user-friendly and performant.
By simplifying certain parts of the visualization, we improve the interpretability of the visualized data.

This visualization tool became a part of the Charles Explorer application and is now available to the users.
