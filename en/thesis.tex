%%% The main file. It contains definitions of basic parameters and includes all other parts.

%% Settings for single-side (simplex) printing
% Margins: left 40mm, right 25mm, top and bottom 25mm
% (but beware, LaTeX adds 1in implicitly)
\documentclass[12pt,a4paper]{report}
\setlength\textwidth{145mm}
\setlength\textheight{247mm}
\setlength\oddsidemargin{15mm}
\setlength\evensidemargin{15mm}
\setlength\topmargin{0mm}
\setlength\headsep{0mm}
\setlength\headheight{0mm}
% \openright makes the following text appear on a right-hand page
\let\openright=\clearpage

%% Settings for two-sided (duplex) printing
% \documentclass[12pt,a4paper,twoside,openright]{report}
% \setlength\textwidth{145mm}
% \setlength\textheight{247mm}
% \setlength\oddsidemargin{14.2mm}
% \setlength\evensidemargin{0mm}
% \setlength\topmargin{0mm}
% \setlength\headsep{0mm}
% \setlength\headheight{0mm}
% \let\openright=\cleardoublepage

%% Generate PDF/A-2u
\usepackage[a-2u]{pdfx}

%% Character encoding: usually latin2, cp1250 or utf8:
\usepackage[utf8]{inputenc}

%% Prefer Latin Modern fonts
\usepackage{lmodern}

%% Further useful packages (included in most LaTeX distributions)
\usepackage{amsmath}        % extensions for typesetting of math
\usepackage{amssymb}
\usepackage{amsfonts}       % math fonts
\usepackage{amsthm}         % theorems, definitions, etc.
\usepackage{bm}             % boldface symbols (\bm)
\usepackage{graphicx}       % embedding of pictures
\usepackage{fancyvrb}       % improved verbatim environment
\usepackage{natbib}         % citation style AUTHOR (YEAR), or AUTHOR [NUMBER]
\usepackage[nottoc]{tocbibind} % makes sure that bibliography and the lists
			    % of figures/tables are included in the table
			    % of contents
\usepackage{dcolumn}        % improved alignment of table columns
\usepackage{booktabs}       % improved horizontal lines in tables
\usepackage{paralist}       % improved enumerate and itemize
\usepackage{xcolor}         % typesetting in color
\usepackage{parskip}		% indentation and space between paragraphs
\usepackage{caption}		% caption formatting
\usepackage[T1]{fontenc}	% font encoding
\usepackage[most]{tcolorbox}
\usepackage{fancyvrb}
\usepackage{relsize}
\usepackage{algpseudocode}
\usepackage{acro}

\newcommand\verbbf[1]{\textbf{#1}}
\newcommand\verbun[1]{\underline{#1}}
\newcommand\verbit[1]{\textit{#1}}
\newcommand\verbxs[1]{\small{\small{#1}}}

\tcbuselibrary{skins,breakable}
\newtcolorbox{mybox}[2][]{breakable,sharp corners, skin=enhancedmiddle jigsaw,parbox=false,
boxrule=0mm,leftrule=2mm,boxsep=0mm,arc=0mm,outer arc=0mm,attach title to upper,
after title={\ }, coltitle=black,colback=gray!10,colframe=black, title={#2},
fonttitle=\bfseries,#1}

%%% Basic information on the thesis

% Thesis title in English (exactly as in the formal assignment)
\def\ThesisTitle{Social network analysis in academic environment}

% Author of the thesis
\def\ThesisAuthor{Jindřich Bär}

% Year when the thesis is submitted
\def\YearSubmitted{2024}

% Name of the department or institute, where the work was officially assigned
% (according to the Organizational Structure of MFF UK in English,
% or a full name of a department outside MFF)
\def\Department{Department of Software Engineering}

% Is it a department (katedra), or an institute (ústav)?
\def\DeptType{Department}

% Thesis supervisor: name, surname and titles
\def\Supervisor{prof. RNDr. Tomáš Skopal, Ph.D.}

% Supervisor's department (again according to Organizational structure of MFF)
\def\SupervisorsDepartment{Department of Software Engineering}

% Study programme and specialization
\def\StudyProgramme{Computer Science}
\def\StudyBranch{Software and Data Engineering}

% An optional dedication: you can thank whomever you wish (your supervisor,
% consultant, a person who lent the software, etc.)
\def\Dedication{%
I would like to express my gratitude to prof. RNDr. Tomáš Skopal, PhD., for his scholarly leadership 
and for the time and advice he has given me during the writing of this thesis. 
I would also like to extend my thanks to my family and friends, who have supported and encouraged me 
greatly during my studies.
}

% Abstract (recommended length around 80-200 words; this is not a copy of your thesis assignment!)
\def\Abstract{
While university information systems usually have detailed data about courses, publication and lecturers, 
they rarely mine the relationships between these entities to provide additional value to the users.
In this thesis, we aim to improve the Charles Explorer web application by utilizing the synthesized academic social network from the existing relational data.
We propose a pipeline for transforming relational data from the university systems into a graph representation of the academic social network. 
We explore the network and propose various ways of utilizing the graph model for inferring missing data. 
Later, we benchmark different re-ranking strategies using the social network metrics against existing academic search engines and show that the social network-based re-ranking can improve the search results ranking.
Lastly, we reimplement the tool for visualising the academic social network in the Charles Explorer application for better user experience.
}

% 3 to 5 keywords (recommended), each enclosed in curly braces
\def\Keywords{%
{social network,} {academia,} {information retrieval,} {reranking}
}

%% The hyperref package for clickable links in PDF and also for storing
%% metadata to PDF (including the table of contents).
%% Most settings are pre-set by the pdfx package.
\hypersetup{unicode}
\hypersetup{breaklinks=true}

% Definitions of macros (see description inside)
\include{macros}
\acsetup{
	make-links = true,
}
\DeclareAcronym{GNN}{
	short = GNN,
	long = graph neural network
}
\DeclareAcronym{CUNI}{
	short = CUNI,
	long = Charles University
}
\DeclareAcronym{OBD}{
	short = OBD,
	long = Osobní bibliografická databáze (Personal Bibliographic Database)
}
\DeclareAcronym{ÚVT}{
	short = ÚVT,
	long = Ústav výpočetní techniky (Computer Science Centre)
}
\DeclareAcronym{DOI}{
	short = DOI,
	first-style = short,
	long = Digital Object Identifier

}
\DeclareAcronym{ISBN}{
	short = ISBN,
	first-style = short,
	long = International Standard Book Number
}
\DeclareAcronym{UKČO}{
	short = UKČO,
	first-style = short,
	long = Číslo osoby (Person ID in the CU information system)
}
\DeclareAcronym{NFD}
{
	short = NFD,
	first-style = short,
	long = Normalization Form D
}
\DeclareAcronym{ASEO}
{
	short = ASEO,
	long = Academic search engine optimization
}


% Title page and various mandatory informational pages
\begin{document}
\include{title}

\renewcommand*{\UrlFont}{\smaller\relax}

%%% A page with automatically generated table of contents of the master thesis

\tableofcontents

%%% Each chapter is kept in a separate file
\chapter*{Introduction}
\addcontentsline{toc}{chapter}{Introduction}

A \textbf{social network} is a theoretical construct describing relations between entities that are usually homogenous in nature.
This definition formed over years mostly from social sciences and psychology, where it was used to create an abstraction of the real-world social interactions between people.
In those fields, the social networks were used to study - among others - social interactions between pairs of people and their influence on the overall structure of the society.

Later on, the social network theory slowly permeated into other fields of study, including discrete mathematics and computer science.
The computing performance of modern computers and the theoretical advances in the field of graph theory allowed for the analysis of large datasets, which in turn allowed for the study of social networks on a larger scale.
This led to the creation of the field of \textit{social network analysis}.

While the theory behind social network analysis is already quite mature, its practical applications are still being explored.
The most prominent use cases include the analysis of social media platforms, where the social network is formed by the users of the platform and their interactions.
Unfortunately, perhaps to keep the competitive advantage, the social media platforms seldom share any details about their social network analysis methods.

Without the access to the user data of commercial social media platforms, we are left with the analysis of social networks that are publicly available.
One such example is the social network of academic researchers - individuals - and their publications - which represent their interactions between each other.
The realm of academic researchers and collaboration on different publications opens a multitude of opportunities for research - since both the ``nodes'' (the researchers) and the ``edges'' (publications) of the social network have interesting data attributes available for them. 

For the researchers, this can be e.g. affiliation with parts of the university, their academic title or their role within the university - e.g. are they lecturers, postdoc researchers, or graduate students helping out on one or two publications etc.

For the publications, the data can include the authors, the publication's affiliation with faculties, year of publishing, publication keywords or a ``topic'' - a categorical variable grouping multiple publications regarding the same topic. This can be inferred from the name and the abstract of the publication using NLP methods.

This thesis demonstrates practical use of social network theory in the context of academic research, a search for collaborators and topic discovery and explores the usability of the social network metrics for re-ranking of document retrieval results.

Lately, the term ``social network'' has been popularized as a synonym to the term \textbf{social media}.
These are online platforms that allow users to create a profile, share content and interact with other users.
Surprisingly enough, this overloading of the term is quite universal between languages - e.g. the Czech expression ``sociální síť'' and the German ``Soziales Netzwerk'' are used for both the social network theory and the social media platforms.
In the rest of this thesis, the term ``social network'' will be used to refer strictly to the theoretical concept or it's concrete representation in the form of a graph, while the term ``social media'' will be preferred for the online platforms.

\section*{Related work}
\addcontentsline{toc}{section}{Related work}

The social network analysis of research groups has been a topic of interest for various publications. 
\cite{ORDOOBADI2019S164} and \cite{CIMENLER2014667} explore the social network of researchers and their publications, and they try to infer the 
social network structure from the data about the co-authorship of the publications. 
Later on they try to compare the social network metrics with the academic performance - citation counts - of the researchers.

\cite{inproceedings} explore the use of social network data for improving the search result ranking in the context of information retrieval.
By using data from 

\section*{Goals of the thesis}
\addcontentsline{toc}{section}{Goals of the thesis}


\chapter{Definitions and notation}

The initial chapter lays out the definitions of the most important social network related terms - graph-related lingo, ego networks, homophily, betweenness, closeness etc. and what all these descibe in general-purpose social networks.

\dots
\chapter{Data models and transformations}

The data about the academic researchers and their publications at CUNI is internally stored in a relational database. 
The first chapter of this thesis explores the transformation of the relational data model into a graph data model, 
which will allow us to use the graph algorithms for the social network analysis.

We also address pitfalls and challenges of the transformation stemming 
from the specific nature of the data and the technical limitations of the source systems.

\section{Input data format}

The Charles University information system is composed of a set of separate systems and applications.

The \textit{Studium} information system\footnote{Available at \url{https://is.cuni.cz/studium/}.} contains the data about the researchers, teachers, the courses and study programmes available at the university.

The \textit{Věda} information system\footnote{Available at \url{https://is.cuni.cz/veda/}, login only.} aggregates the data about creative activities of the researchers, the research projects and inter-university mobility programmes.
For the purpose of this thesis, we are mostly interested in the OBD (or Verso) module, containing the data about the academic publications.

The \textit{Whois} staff information system\footnote{Available at \url{https://is.cuni.cz/webapps/whois2}.} contains the data about the employees of the university, 
their affiliations with the faculties and departments and their academic titles.

While none of the systems offer public APIs, a data import pipeline has been set up by the university's IT department (ÚVT) in advance
for the purpose of the Charles Explorer application. 
This pipeline consists of a set of database views tracking the changes in the data and a script\footnote{\url{https://gitlab.mff.cuni.cz/barj/charles-explorer/-/blob/master/scripts/export.sh}} that exports the data to an SQLite database.

\newpage

\subsection{Exploring the schema}

To illustrate the structure of the data, we provide an example of the schema of the relational database together with example data.
Since this thesis focuses on the social network between the academic researchers and their publications, we limit this example only on the relevant parts of the schema.

The social network-relevant data accessible in the following three database views:

\begin{Verbatim}[commandchars=\\\{\}]
PERSON (
    \verbun{\verbbf{PERSON_ID}}, \verbxs{- UKČO personal number}
    PERSON_NAME, \verbxs{- Full name of the person, incl. the academic titles}
    PERSON_WEBSITE, \verbxs{- Personal website of the person}
    \verbun{\verbbf{PERSON_WHOIS_ID}}, \verbxs{- ID of the person in the Whois system}
    TYPE \verbxs{- Person type - teacher(U), external employee(E), or other(O)}
)
\end{Verbatim}

The \texttt{Person} view contains the data about the academic researchers and teachers at the university.
While the \texttt{TYPE} column might suggest that the table contains records about external people as well (e.g. guest co-authors of publications published by the CUNI researchers),
it only contains the data about the people affiliated with the university. The \texttt{TYPE} column is only used to distinguish between the different employment types at the university.

\begin{Verbatim}[commandchars=\\\{\}]
PUBLICATION_KEYWORDS (
    \verbun{\verbbf{PUBLICATION_ID}}, \verbxs{- internal ID of the publication}
    PUB_YEAR, \verbxs{- year of the publication}
    TITLE, \verbxs{- title of the publication}
    ABSTRACT, \verbxs{- abstract of the publication}
    KEYWORDS, \verbxs{- keywords of the publication}
    LANGUAGUE\verbxs{(sic!)}, \verbxs{- ISO 639-2 language code of the TITLE,}
                              \verbxs{ABSTRACT and KEYWORD columns}
    ORIGINAL\verbxs{ - whether the LANGUAGUE column is the original}
                               \verbxs{language of the publication}
)
\end{Verbatim}

This schema shows some more issues with the data - the \texttt{PUBLICATION\_KEYWORDS} is missing a single primary key attribute,
as the \texttt{PUBLICATION\_ID} is not unique across the table. This is because the same publication can have multiple records in the table,
each record representing a different language version of the title, abstract and keywords.

Moreover, the \texttt{PUBLICATION\_KEYWORDS} view does not contain an universally accepted publication identifier (e.g. a DOI or an ISBN) 
that would allow us to link the publication to the external sources of the publication data.
This is because of a technical limitation of the information systems, as the \textit{OBD} module is not fully interoperable with the \textit{Studium} module we are 
consuming the data from.

\newpage

\label{sec:pub-author-all}
\begin{Verbatim}[commandchars=\\\{\}]
PUBLICATION_AUTHOR_ALL (
    \verbun{PUBLICATION_ID}, \verbxs{- internal ID of the publication}
    \verbun{PERSON_ID}, \verbxs{- UKČO personal number of the author, if available}
    PERSON_NAME, \verbxs{- Full name of the author, incl. the academic titles}
)
\end{Verbatim}

The relational database view \texttt{PUBLICATION\_AUTHOR\_ALL} contains the links between the publications and the authors from the previous views.

With the most naive approach, the transformation of the relational data model into a graph data model could be done by transforming the records of the 
\texttt{PERSON} and (deduplicated) \texttt{PUBLICATION\_KEYWORDS} views into the nodes of the graph, 
and the records of the \texttt{PUBLICATION\_AUTHOR\_ALL} view into the edges of the graph.

\section{Transformation}

In the \hyperref[sec:pub-author-all]{comment} for the \texttt{PUBLICATION\_AUTHOR\_ALL.PERSON\_ID} column, we note that the UKČO personal number is not always available.
This is because of external authors, who are not affiliated with the university and do not have a UKČO personal number.
What is worse, such authors are only identified by their name and academic titles, which can be inconsistent across the publications.
This is again caused by the technical limitations of the source systems, as the interoperability between the \textit{OBD} and \textit{Studium} modules is limited.

This poses a challenge for the transformation of the relational data model into a graph data model, as we need to come up 
with a way to identify the external authors and link them to the publications.

\subsection{Inferring missing identities}

As mentioned above, the available relational data model is unfit for a direct transformation into a graph data model.
This is because the data about the external authors is incomplete and inconsistent, which might cause some of the authors
to be represented by multiple nodes in the graph.

Aside from the obvious implications of this issue - i.e. user confusion and potential performance issues,
this also poses a challenge for the social network analysis, as the graph algorithms might not be able 
to correctly identify the external authors.

In the aforementioned \texttt{PUBLICATION\_AUTHOR\_ALL} view, counts of the relations mentioning the authors with / without the UKČO personal number are as follows:

\begin{figure}[!ht]
    \captionsetup{width=.9\linewidth}
    \centering
    \begin{tabular}{|c|c|c|}
    \hline
        Type & Count & Distinct \\ \hline
        PERSON\_ID present & 808467 & 39523 \\ \hline
        No PERSON\_ID & 671332 & ? \\ \hline
    \end{tabular}
    \caption{Counts of the relations in the \texttt{PUBLICATION\_AUTHOR\_ALL} view.}
\end{figure}

We see that the no-identifier authors take up to 45\% of the total count of the relations.
This explains the importance of the problem of inferring the missing identities.

\subsubsection{Naïve approach}

The naïve solution to this problem is using the \textit{names} for the identity inference 
in case of missing identifiers.
While this is a simple and straightforward transformation, it has obvious drawbacks.

Firstly, the names are not guaranteed to be \textit{unique} across the dataset. 

Secondly, the names are not guaranteed to be \textit{consistent} across the dataset - due to name changes (e.g. marital), 
typos, or different conventions in the academic titles.
See the example of a search for the name ``Jaroslav Peška'' in the \texttt{PUBLICATION\_AUTHOR\_ALL} view:

\begin{figure}[!ht]\label{fig:jaroslav-peska}
    \captionsetup{width=.9\linewidth}
    \centering
    \begin{tabular}{|c|c|c|}
    \hline
        COUNT(*) & PERSON\_NAME & PERSON\_ID \\ \hline
        2 & doc. PhDr. Jaroslav Peška Ph.D. & 14124313\dots \footnotemark \\ \hline
        5 & doc. PhDr. Jaroslav Peška Ph.D. & null \\ \hline
        2 & Doc. PhDr. Jaroslav Peška Ph.D. & null \\ \hline
        1 & doc. PhDr. Jaroslav Peška PhD. & null \\ \hline
        4 & Jaroslav Peška & null \\ \hline
    \end{tabular}
    \caption{Search for the name ``Jaroslav Peška'' in the \texttt{PUBLICATION\_AUTHOR\_ALL} view.}
\end{figure}

\footnotetext{PERSON\_ID redacted for privacy reasons, replaced by truncated PERSON\_WHOIS\_ID.}

While there are 5 variants of the same name in the dataset, only one of them is correctly linked to the UKČO personal number.
This could have been caused by human error on data input or by the limitations of the source systems. 
Either way, this creates an unrecoverable loss of information in the dataset.

In the SQL database export, we can ``merge'' the records using the naïve approach efficiently using a many-to-one (non-injective) mapping
on the \texttt{PERSON\_NAME} column.

Let us define a function $f$:
$$f: \text{PERSON\_NAME} \to \text{NORMALIZED\_NAME}$$

We require $f$ to map all person names to a \textit{normalized form}, which is defined as follows:
\begin{itemize}
    \item The academic titles are stripped from the name.
    \item The name is converted to lowercase.
    \item The name is stripped of any diacritics.
    \item The name is stripped of any non-alphabetic characters.
    \item The whitespace characters are normalized to a single space.
    \item The name is stripped of any leading or trailing whitespace.
\end{itemize}

We can see that in the case of \hyperref[fig:jaroslav-peska]{Jaroslav Peška}, 
the normalized version for all the variants is the same (i.e. $f(\texttt{PERSON\_NAME}) = \texttt{jaroslav peska}$)
and the records have been effectively merged.

% \section{Processing available data}

% \subsection{Categorizing publications}

% This section talks about analyzing the publication data using different NLP techniques and categorizing them into both predefined and inferred categories using classic algorithms (for example K-means and the alternatives) and ML approaches (Bayesian topic modelling using LDA).

% This is a part of the data mining done on the data for enhancing the graph data model. As of now, the publication data is rather opaque with only the publication identificators, relations to people and text annotations in natural languages. Through the related people, there are also transitive relations to other publications, which most likely share the topic with the original publication. This is also taken as a benchmark data for the NLP-powered categorization tasks.

% \subsubsection{Wordnet}

% \subsubsection{Embedding with count-based techniques (TF-IDF, GloVe)}

% \subsubsection{LLM based embeddings}

% \subsection{Clustering the data}

% \subsubsection{LDA topic modelling}

% \subsubsection{K-means with other embeddings}

% \section{Inferring missing data}

% This section talks about the data inference for missing data. This is a problem mostly for the external authors (with little to no affiliation to CUNI) who can be often only identified by their name and academic titles. We try to come up with a better way of ``filling in the blanks'' - either by using some basic statistics or some data extracted from the social network itself.

% \subsection{Infering identities from the social network}

% Search for Jaroslav Peška yields the following results:
% \begin{itemize}
%     \item doc. PhDr. Jaroslav Peška Ph.D. - 77815713
%     \item doc. PhDr. Jaroslav Peška Ph.D. - null
%     \item Jaroslav Peška - null
%     \item Doc. PhDr. Jaroslav Peška Ph.D. - null
%     \item doc. PhDr. Jaroslav Peška PhD. - null
% \end{itemize}

% Another example is the search for Dmytro Yu Balakin, which yields the following results:

% \begin{itemize}
%     \item Dmytro Yu Balakin
%     \item Dmytro Yu Batakin
%     \item Dmytro Yu. Balakin.
% \end{itemize}

% It is clear that most of those are likely the same person, but the data is not consistent. The first one is the correct one, but the other ones are missing the academic titles and the ``doc.'' prefix. The ``PhD.'' suffix is also inconsistent - sometimes it is ``PhD.'', sometimes ``Ph.D.''. The ``doc.'' prefix is also sometimes missing.

% ...
% ...

% While it might be tempting to solve this "online" by visually merging the entities in the application's visualization layer, this causes multiple issues. The obvious one are the performance implications 

% \section{Analyzing the social network}

% The first contact with the network, description of some basic properties, discovering the features of the data and trying to mine some non-obvious relations from the data.

% \dots
\chapter{Social network boosted search ranking}

Charles Explorer is an academic search engine, which allows the user to search for publications, classes, people and study programmes in the academic domain.

The following sections, we look more into the search engine part of the application, explore the possible issues with the search results ranking, 
and experiment with utilizing the social network data for the search results ranking improvement.

\section{Full-text search}

Full-text search is nowadays an essential part of any information retrieval system. 
Many search engines - including Apache Solr in Charles Explorer - implement the full text search by utilizing the TF-IDF algorithm.
This is a simple and efficient way to rank the search results based on the relevance of the documents to the search query.

The TF-IDF algorithm is based on the term frequency and inverse document frequency of the terms in the documents - a \textit{document} is in general unstructured free text content.
Some entities in our academic search engine map well to this notion of a \textit{document} - e.g. a \texttt{publication} or \texttt{class} both have inherent textual content (titles, abstracts or syllabi).
Unfortunately, this does not hold for all the entities in the system.

As an example, a \texttt{person} entity usually does not have any explicit textual content associated with it. 
When searching for a person interested in a particular topic, the search engine has to rely on the textual content of the publications, classes, 
or other entities associated with the person, i.e. traverse - at least implicitly - the knowledge graph (or the social network of the person).

A simple solution to this problem would be to represent every person as a document, concatenating all the textual content of the entities associated with the person.
This can be further refined by assigning different weights - e.g. to the different types of entities (a \texttt{class} might be more important than a \texttt{publication}), 
or different concatenated parts of the documents (e.g. the publication title and class name are more important than the abstracts and syllabi).

Regarding adding new entities related to a given person, the concatenation also serves us well - we can simply append the new entity's content to the person's document and reindex it.

However, this approach also has several drawbacks. First of all, assuming we're building a general academic search engine allowing for search in publications, classes and people, we would be indexing the same content multiple times.
This is not only inefficient in terms of storage, but also prone to update errors - there is multiple copies of the same content, which have to be updated separately.
The second issue arises from the concatenation - if any of the person's associated entities changes, the whole document has to be reindexed.
However, this might be less of a problem than the first issue, as it's not too common for academic records to get updated or removed - at least in comparison to the number of new records being added.

\subsection{Result ranking in the naive case}

Result ranking in information retrieval refers to the ordering of the search results when presented to the end user. This is often based on the relevance of the documents to the current search query.
The relevance-based ranking is often enough for the basic use case - the user is presented with the most relevant documents first, and can further explore the less relevant ones if needed.

While it might seem a bit superficial, the ranking is in fact still part of the information retrieval process. 
As multiple studies show \footnote{https://link.springer.com/article/10.1007/s11151-014-9435-y}\footnote{https://link.springer.com/article/10.1007/s10791-010-9150-8}, 
the ranking of the search results positively correlates with the click-through rate of the results 
- likely because of the typical top-left to bottom-right reading pattern of the users.
This can be further affected by other, more technical factors - such as the need for an additional user action like scroll or pagination to see the results further down the list.

%% https://link.springer.com/article/10.1007/s10791-010-9150-8 - Re-ranking search results using an additional retrieved list

Considering a simple tf-idf based search engine, the ranking of the search results is based on the relevance of the documents to the search query.
This is directly related to the term frequency of the query in the document - for a fixed query and a given document collection, we can forget about the inverse document frequency, as it's constant over all the documents.
Ranking the documents solely based on the term frequency might however lead to unfavourable results - especially in the case of a proxy-representation of a given entity.


\begin{figure}[ht!]
    \includegraphics[width=0.8\textwidth]{../img/bob-alice-soc-netw.png}
    \centering
    \caption{Simple representation of the social network of Alice and Bob.}
\end{figure}

Let us explore the issues on an example where we represent people as documents, concatenating the textual content of the entities associated with them.
Consider two academic researchers in our system - \textit{prof. Alice} and \textit{Bc. Bob}. 
Bob is a student of Alice, and has published several papers on \textit{Information retrieval} with her. Aside from those, Bob has not published any other papers.
On the other hand, Alice has published a lot of papers on various topics - related to IR, but also to other similar fields. 
See a simple representation of their social network above.

Note that aside from the common publications, Bob has no other entities associated with him, while Alice has other publications with other co-authors.

Because of this, the document representation of Bob will be short, compared to the one of Alice.
Assuming the search query is \textit{Information Retrieval} - and there are no other publications or classes related to this topic in the system - 
the search engine will consider Bob as the more relevant person for this topic. This is becuase the term frequency of the query in Bob's document is higher than in Alice's document - solely because of the length.


\begin{figure}[ht!]
    \includegraphics[width=0.8\textwidth]{../img/bob-alice-representations.png}
    \centering
    \caption{Concatenating representations of Alice and Bob as documents.}
\end{figure}

We see that the document of Bob is shorter than the one of Alice, because he has less associated entities. We can also notice that the Alice's document fully contains the Bob's document.
    
The colored terms represent the current search query - if Alice only published papers on the topic of the query with Bob, the term frequency of the query in Bob's document would be higher.
In case Alice also publishes on the topic with other co-authors, it gets harder to reason about which of the term frequencies is higher.

In either way, this is most likely not the desired result - while Bob has published papers only on the given topic - and therefore, by the TF-IDF measure, is more relevant to the query,
it is likely that Alice is the more relevant person for the user, since she has published more papers in total (perhaps also on the topic of the query).


\section{Re-ranking}

A rather theoretical section about the different ranking methods and reasons for re-ranking. Later on, we propose the actual algorithm for search result re-ranking based on the social network data (and the current search results). 

\dots

\section{Benchmarks}

A section discussing the performance benchmarks for different search result rankings.

\dots

\section{Implementation details, performance}

A section about the implementation details for the social network enhanced search within Charles Explorer. Can contain parts about the actual real-time performance (and discussion whether the added reranking value is worth the potential performance degradation).

\dots


\chapter{Social network visualization}

Even though the data mining processes described in the previous chapters give us valuable insights into the structure of social networks, 
they are not necessarily easy to interpret for laymen.
One way to make the results of data mining more accessible is to create data visualization, i.e. present the data with their visual representation, 
while using different visual cues to guide the viewers' attention towards different qualities.

In this part of the thesis, we will assess the current state of the visualizations in the Charles Explorer application, 
will propose some improvements to make the visualization more accessible to the users and will implement those.

\section{Vocabulary}

This section introduces some basic vocabulary that will be used throughout this chapter.

\textbf{Ego network}: Ego-centric networks (or shortened to “ego” networks) are a particular type of network which specifically maps the connections of and from the perspective of a single person (an “ego”). (\cite{Lizardo2020-xo}).

\textbf{Visual decoding}: Also called \textit{preattentive processing} or \textit{preattentive vision}, the visual decoding is the instantaneous perception of the visual field that comes without apparent mental effort. (\cite{Cleveland1985})

\section{Assessing the current state}

The current state of the Charles Explorer visualization views is quite simple. 

In the \textit{Person} search mode, the user can search for people inside the Charles University. 
When accessing a person's profile, the application shows the person's \textit{ego network} with the main person and their direct collaborators 
as nodes and their common publications aggregated to the edges.

\begin{figure}[ht!]
    \includegraphics[width=0.8\textwidth]{../img/charles-explorer-old-view.png}
    \centering
    \caption{Charles Explorer showing the \textit{ego network} of a person.}
\end{figure}

The graph is displayed with force-directed layout. The edge thickness is proportional to the number of common publications between 
the two people and the colors of the nodes represent the person's faculty association.

\subsection{Problems with color coding}

This approach has multiple drawbacks. 
Firstly, the color coding of the nodes does not prove very useful, as it hinders the visual decoding of the graph, 
since the user has to spend more attention on reading the legend, rather than interpreting the graph subconciously.

This is especially true for larger ego networks with many nodes with different faculty affiliations. 
Additionaly, the application does not provide any alternative visual cue for color vision deficient users.

According to \cite{Cleveland1985}, the discriminability of colors is limited to 5 - 6 colors, which is not enough for the 17 faculties and departments 
of the Charles University. With faculties, there is also little room for aggregation (of multiple faculties into one color), as the faculties are not 
hierarchical.

\subsection{Layouting problems}

The arbitrary positions of the nodes - based on the physical simulation layout - increase the cognitive load of the viewer and contribute to the graph's worse readability.

\subsection{Other visualization problems}

\section{Graph layouting algorithms}

Before we start, let us introduce some basic terminology that will be used throughout this chapter.

A \emph{graph layout} is a way to position the nodes of a graph in a two-dimensional space, such as on a screen or in print. 
This is a necessary step of any graph data visualisation task - without computing the layout, the graph nodes nor edges do not have any intrinsic location assigned to them.

There are several various ways of computing the graph layout, which we'll describe now in no particular order:

\begin{itemize}
    \item \textbf{Hierarchical layout} - for certain types of graphs, it's beneficial to display them in a hierarchical manner. 
    This is usually the case for trees, i.e. connected acyclic graphs - such as family trees or organisation structures.

    Such layout is typically laid out on either horizontal or vertical axis, with the root node at the top or leftmost position, and the children nodes below or to the right of their parent nodes.
    The other direction (i.e. vertical for horizontal-major graphs and vice versa) is used for layouting the incomparable sibling groups.
    
    While this layout might be used on cyclic graphs as well, it's not as common, as it's not as easy to interpret for laymen as the other layouts.
    It might be useful for graphs with some kind of hierarchy or other clear ``score'' quality, such as social networks with a clear leader, or a company structure.

    \item \textbf{Circular layout} - this layout places the nodes on a circle, with the edges connecting them. 
    This layout is useful for visualizing relations between members of a group or community, as it's easy to see which nodes are connected to which other nodes.
    Some work can be done for calculating the optimal placement of nodes - be it minimizing the edge length, or minimizing the number of edge crossings.

    While minimizing the number of edge crossings might help with the graph readability, it might result in a less visually appealing graph, as the nodes might be placed further apart than necessary.
    % TODO: add a reference to the paper about this.

    Trying to minimize the edge length by ordering the nodes on the circle also results in a more readable graph, as this layout promotes grouping closely connected subcommunities together.

    In some cases, the global position of nodes in the layout can depict some qualitative variable - e.g. in a social network of people from different social groups, 
    the node position on the y-axis might represent the person's annual income. With the edges between the nodes representing the social connections,
    such layout can help us understand the relation between the social status and the connections between the people.

    \item \textbf{Grid layout} - this layout places the nodes on a grid, with the edges connecting them.
    While some optimizations, such as minimizing the edge length can be done in this case as well, 
    this type of layout usually results in a less readable graph.

    An equally-spaced grid layout only communicates the relations between nodes by drawing the edges between them, not utilising the position of the nodes.
    It is often the spatial proximity of the nodes that helps viewers understand the relations between the nodes subconciously - and while we still can 
    supply this information by using different visual cues (e.g. coloring of the nodes), it's often not as intuitive as the spatial proximity.

    Unlike the hierarchical and circular layouts, there does not seem to be a clear use for the global node position for depicting some qualitative variable.

    In some applications, the grid layout is sometimes used as the initial graph layout for the force-directed layout, as it's easier to calculate the initial positions of the nodes on a grid.

    \item \textbf{Force-directed layout} - this layout is based on the physical simulation of the forces between the nodes and edges of the graph.
    The nodes are treated as charged particles, and the edges as springs - the nodes repel each other, while the edges try to keep the nodes connected.
    
    This results in a layout where the nodes are placed in such a way that the edges are as short as possible, while the nodes are as far apart as possible.
    This is useful for visualizing the overall structure of the graph, as it's easy to see which nodes are connected to which other nodes, and how the graph is connected.

    Using the edges to guide the layouting process results in a more ``intuitively'' readable graph, as the edges representing some sort of relation between two nodes 
    cause the nodes to be drawn to each other. This means that related nodes are layouted closer to each other - therefore supporting the spatial proximity Gestalt principle.

    The force-directed layout is especially useful for visualizing large social networks, as it's easy to see which nodes are connected to which other nodes, and how the graph is connected.
    Such graph is also easy to navigate, and the viewer can easily find bridges, hubs, or other important features of the graph.

    Similar to the grid layout, the global node position cannot be used to depict qualitative variables. While it might be technically possible 
    to add auxiliary forces to the simulation to achieve this - e.g. pull nodes to the left or right of the screen based on some variable - 
    it's not as common as in the hierarchical or circular layouts. This approach might also interfere with the edge-based layouting process,
    causing the graph to be less readable.

    Because of this reason and the very essence of the force-directed layouts, the location of the nodes in such layout is quite arbitrary 
    - in case of large graphs, the viewer might not be able to find the node they are looking for. This is why it's often useful to add 
    some sort of search functionality or other navigation aids to the visualization.
\end{itemize}


\chapter*{Conclusion}
\addcontentsline{toc}{chapter}{Conclusion}

The conclusion wraps up the thesis, describing the achievements we've reached and suggests ideas for further research.

\dots

%% Bibliography
\include{bibliography}

%%% Figures used in the thesis (consider if this is needed)
% \listoffigures

%%% Tables used in the thesis (consider if this is needed)
%%% In mathematical theses, it could be better to move the list of tables to the beginning of the thesis.
% \listoftables

%%% Abbreviations used in the thesis, if any, including their explanation
%%% In mathematical theses, it could be better to move the list of abbreviations to the beginning of the thesis.
\chapwithtoc{List of Abbreviations}
\printacronyms[
	heading=none,
	display=used,
	sort=true,
]


%%% Attachments to the master thesis, if any. Each attachment must be
%%% referred to at least once from the text of the thesis. Attachments
%%% are numbered.
%%%
%%% The printed version should preferably contain attachments, which can be
%%% read (additional tables and charts, supplementary text, examples of
%%% program output, etc.). The electronic version is more suited for attachments
%%% which will likely be used in an electronic form rather than read (program
%%% source code, data files, interactive charts, etc.). Electronic attachments
%%% should be uploaded to SIS and optionally also included in the thesis on a~CD/DVD.
%%% Allowed file formats are specified in provision of the rector no. 72/2017.

% \appendix
% \chapter{Attachments}

% \section{First Attachment}

\openright
\end{document}
