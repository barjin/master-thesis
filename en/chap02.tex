\chapter{Data models and transformations}

This chapter talks about the way the format the data is received in (a relational database) and the differences between a relational and a graph model. In the following sections, we figure out how to effectively transform the relational data into a social network and what additional information does the transformed model tell us about the data (and how to get/calculate the information, again, effectively).  

\dots

\section{Relational vs. graph model}

A short theoretical section about the differences between the two different data models.

\dots

\section{Transformation}

A section about the details of the relational - graph transformation, the way the graph is stored (and reasons why - with respect to the operations we want to run on the data).

\dots

\section{Inferring missing data}

This section talks about the data inference for missing data. This is a problem mostly for the external authors (with little to no affiliation to CUNI) who can be often only identified by their name and academic titles. We try to come up with a better way of ``filling in the blanks'' - either by using some basic statistics or some data extracted from the social network itself.

\section{Analyzing the social network}

The first contact with the network, description of some basic properties, discovering the features of the data and trying to mine some non-obvious relations from the data.

\dots