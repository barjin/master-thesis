\chapter{Definitions and notation}

The initial chapter lays out the definitions of the most important social network related terms and concepts. 
The chapter also introduces the notation used throughout the thesis.

\section{Discrete graphs}

In discrete mathematics, \textit{graph} is a mathematical structure that consists of a set of \textit{nodes} (often denoted as $V$ for \textit{vertices})
and a set of \textit{edges} (often denoted as $E$) that connect pairs of the nodes.

This section introduces some of the less common graph-related terms and concepts that are used in the thesis.

\theoremstyle{definition}
\newtheorem{definition}{Definition}[section]

\begin{definition}[Adjacency matrix]
    The \textit{adjacency matrix} of a graph $G = (V, E)$ is a square matrix $A$ of size $|V| \times |V|$ where $A_{uv} = 1$ if there is an edge between nodes $u$ and $v$, and $A_{uv} = 0$ otherwise.  

    Different matrix operations can be used to calculate various properties of the graph - e.g. the degree of a node can be calculated as the sum of the elements in the row of the adjacency matrix corresponding to the node.
\end{definition}

\begin{definition}[Distance matrix]
    The \textit{distance matrix} of a graph $G = (V, E)$ is a square matrix $D$ of size $|V| \times |V|$ where $D_{uv}$ is the length of the shortest path between nodes $u$ and $v$.

    The distance matrix can be calculated using algorithms such as Floyd-Warshall algorithm or repeated Dijkstra's algorithm.
\end{definition}

\begin{definition}[Node neighborhood]
    The \textit{neighborhood} of a node $v$ in a graph $G$ is the set of all nodes that are connected to $v$ by an edge.
    $$
    N(v) = \{ u \in V \mid \text{there exists an edge between $v$ and $u$} \}
    $$
\end{definition}

\begin{definition}[k-hop neighborhood]
    The \textit{k-hop neighborhood} of a node $v$ in a graph $G$ is the set of all nodes that are reachable from $v$ by traversing at most $k$ edges.

    $$
    N_k(v) = \{ u \in V \mid \text{there exists a path of length at most $k$ from $v$ to $u$} \}
    $$

    This can be useful to capture the local structure of a graph 
    - e.g. \cite{nikolentzos2019khop} use the concept of k-hop neighborhoods to enhance graph embeddings in \ac{GNN}.
\end{definition}

\begin{definition}[Subgraph]
    A \textit{subgraph} $G' = (V', E')$ of a graph $G = (V, E)$ is a graph where $V' \subseteq V$ and $E' \subseteq E$.
\end{definition}

\begin{definition}[Induced subgraph]
    An \textit{induced subgraph} $G' = (V', E')$ of a graph $G = (V, E)$ is a subgraph where $V' \subseteq V$ and $E' = \{ (u, v) \in E \mid u, v \in V' \}$.

    In simpler terms, an \textit{induced subgraph} is a subgraph that contains all the edges between the nodes in the subgraph.
\end{definition}

\begin{definition}[Bipartite graph]
    A \textit{bipartite graph} is a graph where the nodes can be divided into two disjoint sets $V_1$ and $V_2$ such that all edges connect nodes from $V_1$ to nodes from $V_2$.
    
    Formally, a graph $G = (V, E)$ is bipartite if there exists a partition of $V$ into two sets $V_1$ and $V_2$ such that
    $$E \subseteq \{ (u, v) \mid u \in V_1, v \in V_2 \}$$
\end{definition}

\begin{definition}[Monopartite projection]\label{def:monopartite-projection}
    The \textit{monopartite projection} of a bipartite graph $G = (V_1 \cup V_2, E)$ onto a set of nodes $V_1$ is a graph $G' = (V_1, E')$, 
    where the nodes are the nodes in $V_1$ and the edges are between the nodes in $V_1$ that share a common neighbor in $V_2$.
    
    Formally defined, the monopartite projection is a graph $G' = (V_1, E')$ where
    $$
    E' = \{ (u, v) \mid \exists w \in V_2 \text{ such that } (u, w) \in E \text{ and } (v, w) \in E \}
    $$

    Projection algorithms can produce multiple edges between the nodes in the projection.
    Resolution of this problem largely depends on the specific use case of the projection - e.g. $G'$ can be a multigraph or the edges can be weighted by the number of common neighbors from $V_2$.
    
\end{definition}
\section{Social networks}

While the social sciences have been studying social networks for decades, not all of the terminology is relevant for the purposes of this thesis.
This section introduces the most important computation-related social network terms and concepts that are used in the thesis.

For the purposes of this thesis, a \textit{social network} is a graph where the nodes represent real-world entities (people, their publications) and the edges represent relationships between the entities (only authorship in this thesis).

\begin{definition}[Ego network]
    The \textit{ego network} of a node $v$ in a social network $G$ is the induced subgraph of $G$ that contains $v$ and its (1-hop) neighborhood.
    $$
    E(v) = (N(v), \{ (v, u) \mid u \in N(v) \})
    $$
\end{definition}

\begin{definition}[Ego (vertex)]
    The \textit{ego} of a node $v$ in an ego network $E(v)$ is the node $v$ itself.
\end{definition}

\begin{definition}[Alter (vertex)]
    An \textit{alter} of a node $v$ in an ego network $E(v)$ is any node $u$ that is not the ego $v$.
\end{definition}

\subsection{Centrality measures}

In social network analysis, \textit{centrality measures} are used to quantify the importance of nodes in a social network.
This section introduces the most common centrality measures - and the related topics - that are used in the thesis.

\begin{definition}[Node degree]
    \label{def:node-degree}
    The \textit{degree} of a node $v$ in a graph $G$ is the number of edges incident to $v$.
    $$
    \text{deg}(v) = |N(v)|
    $$

    The \textit{degree centrality} of a node $v$ is the degree of $v$ normalized by the maximum possible degree in the graph.
\end{definition}

\begin{definition}[Betweenness centrality]
    The \textit{betweenness centrality} of a node $v$ in a graph $G$ is the fraction of all shortest paths that pass through $v$.
    $$
    \text{betw}(v) = \sum_{s \neq v \neq t} \frac{\sigma_{st}(v)}{\sigma_{st}}
    $$

    where $\sigma_{st}$ is the number of shortest paths between nodes $s$ and $t$, and $\sigma_{st}(v)$ is the number of those paths that pass through $v$.
\end{definition}

\begin{definition}[Eigenvector centrality]
    The \textit{eigenvector centrality} of a node $v_i$ in a graph $G$ is the sum of the centrality scores of the nodes that are connected to $v$.
    $$
    \text{eig}(v_i) = \frac{1}{\lambda} \sum_{v_j \in N(v)} \text{eig}(v_j)
    $$

    where $\lambda$ is a constant. Note that the definition can also be rewritten as 
    $$
    \text{eig}(v_i) = \frac{1}{\lambda} \sum_{v_j \in V} a_{v_i, v_j} \text{eig}(v_i)
    $$

    with $a_{vu}$ being the elements of the adjacency matrix of the graph $G$.
    
    Denoting $\mathbf{eig(v)}$ as a vector of eigenvector centralities for all nodes in the graph, the equation can be rewritten as

    $$
    \lambda \; \mathbf{eig(v)} = A \; \mathbf{eig(v)}
    $$

    where $A$ is the adjacency matrix of the graph $G$. 
    This shows the reasoning behind the name of the centrality measure - the centrality scores are the eigenvectors of the adjacency matrix.
\end{definition}

\begin{definition}[Katz centrality]
    The \textit{Katz centrality} of a node $v_i$ in a graph $G$ is defined as
    $$
    \text{katz}(v_i) = \sum_{k=1}^{\infty} \sum_{j=1}^n \alpha^k (A^k)_{ij}
    $$
    where $\alpha \in (0, 1/\lambda_{\text{max}})$ is a constant, and $\lambda_{\text{max}}$ is the largest eigenvalue of the adjacency matrix $A$.

    The Katz centrality counts the number of paths between the central node and all other nodes in the graph.
    For a path of length $k$ is additionally discounted by the \textit{attenuation} factor $\alpha^k$.

    The above formula uses the fact that the power of the adjacency matrix $A^k$ counts the number of paths of length $k$ between nodes $i$ and $j$.
\end{definition}

\begin{mybox}{}
As follows from the definitions, the betweenness centrality of a node $v$ is computationally expensive 
to calculate for large graphs, as it considers all pairs of nodes in the graph.

While e.g. \cite{brandes-faster-centrality} offers an algorithm that reduces the computational complexity of the betweenness centrality calculation,
the calculation still poses a significant performance bottleneck for large graphs.

On the other hand, the Katz centrality can be calculated more efficiently - since the definition introduces 
the attenuation factor $\alpha$, the importance of the nodes that are further away from the central node is discounted
and the centrality measure can be approximated by truncating the sum at a certain small value of $k$.
\end{mybox}

\section{Classifier evaluation metrics}

In the context of this thesis, the social network metrics are used to evaluate 
the performance of the classifiers that are run on the social network data.

This section introduces the most common classifier evaluation metrics that are used in the thesis.

\begin{definition}[Precision]
    The \textit{precision} of a classifier is the ratio of the true positive predictions to the total number of positive predictions.
    $$
    \text{precision} = \frac{\text{TP}}{\text{TP} + \text{FP}}
    $$
\end{definition}

\begin{definition}[Recall]
    The \textit{recall} of a classifier is the ratio of the true positive predictions to the total number of actual positive instances.
    $$
    \text{recall} = \frac{\text{TP}}{\text{TP} + \text{FN}}
    $$
\end{definition}

\begin{definition}[F1 score]
    The \textit{F1 score} of a classifier is the harmonic mean of the precision and recall.
    $$
    \text{F1} = 2 \cdot \frac{\text{precision} \cdot \text{recall}}{\text{precision} + \text{recall}}
    $$
\end{definition}

\begin{definition}[Macro-averaged F1 score]
    For a multi-class classification task, the \textit{macro-averaged F1 score} is the average of the F1 scores for each class.
    $$
    \text{macro-F1} = \frac{1}{n} \sum_{i=1}^{n} \text{F1}_i
    $$
\end{definition}

\begin{definition}[Micro-averaged F1 score]
    For multi-class classification tasks, the \textit{micro-averaged F1 score} is the F1 score calculated from the total number of true positives, false positives, and false negatives.
    $$
    \text{micro-F1} = 2 \cdot \frac{\sum_{i=1}^{n} \text{TP}_i}{\sum_{i=1}^{n} \text{TP}_i + \sum_{i=1}^{n} \text{FP}_i + \sum_{i=1}^{n} \text{FN}_i}
    $$
\end{definition}