\chapter*{Introduction}
\addcontentsline{toc}{chapter}{Introduction}

A \textbf{social network} is a theoretical construct describing relations between entities that are usually homogenous in nature.
This definition formed over years mostly from social sciences and psychology, where it was used to create an abstraction of the real-world social interactions between people.
In those fields, the social networks were used to study - among others - social interactions between pairs of people and their influence on the overall structure of the society.

Later on, the social network theory slowly permeated into other fields of study, including discrete mathematics and computer science.
The computing performance of modern computers and the theoretical advances in the field of graph theory allowed for the analysis of large datasets, which in turn allowed for the study of social networks on a larger scale.
This led to the creation of the field of \textit{social network analysis}.

While the theory behind social network analysis is already quite mature, its practical applications are still being explored.
The most prominent use cases include the analysis of social media platforms, where the social network is formed by the users of the platform and their interactions.
Unfortunately, perhaps to keep the competitive advantage, the social media platforms seldom share any details about their social network analysis methods.

Without the access to the user data of commercial social media platforms, we are left with the analysis of social networks that are publicly available.
One such example is the social network of academic researchers - individuals - and their publications - which represent their interactions between each other.
The realm of academic researchers and collaboration on different publications opens a multitude of opportunities for research - since both the ``nodes'' (the researchers) and the ``edges'' (publications) of the social network have interesting data attributes available for them. 

For the researchers, this can be e.g. affiliation with parts of the university, their academic title or their role within the university - e.g. are they lecturers, postdoc researchers, or graduate students helping out on one or two publications etc.

For the publications, the data can include the authors, the publication's affiliation with faculties, year of publishing, publication keywords or a ``topic'' - a categorical variable grouping multiple publications regarding the same topic. This can be inferred from the name and the abstract of the publication using NLP methods.

This thesis demonstrates practical use of social network theory in the context of academic research, a search for collaborators and topic discovery and explores the usability of the social network metrics for re-ranking of document retrieval results.

Lately, the term ``social network'' has been popularized as a synonym to the term \textbf{social media}.
These are online platforms that allow users to create a profile, share content and interact with other users.
Surprisingly enough, this overloading of the term is quite universal between languages - e.g. the Czech expression ``sociální síť'' and the German ``Soziales Netzwerk'' are used for both the social network theory and the social media platforms.
In the rest of this thesis, the term ``social network'' will be used to refer strictly to the theoretical concept or it's concrete representation in the form of a graph, while the term ``social media'' will be preferred for the online platforms.

\section*{Related work}
\addcontentsline{toc}{section}{Related work}

This section talks about the related work in the field of social network analysis and its applications in the context of academic research.
It is separated into two parts - while the first subsection talks about the social network analysis in general, 
the second part mentions different academic research result retrieval systems and their features.

\subsection*{Theoretical background}
\addcontentsline{toc}{subsection}{Theoretical background}

The social network analysis of research groups has been a topic of interest for various publications. 
\cite{ORDOOBADI2019S164} and \cite{CIMENLER2014667} explore the social network of researchers and their publications, and they try to infer the 
social network structure from the data about the co-authorship of the publications. 
Later on they try to compare the social network metrics with the academic performance - citation counts - of the researchers.

\cite{social-relevance-for-re-ranking-documents} explore the use of social relevance data for improving the search result ranking in the context of information retrieval.
By using data from the \textit{Umaps Knowtex} dataset (no longer available) they show that the social relevance metrics - like share count, comment count or like count - can be used to improve the ranking of the search results.

\cite{social-model-literature-access} introduce a social model for academic literature access systems, 
where metrics on the social network of researchers, their publications and the users of the system is used to improve 
the search results ranking. In this paper, the authors explore weighted approach with multiple measures like betweeness, closeness or PageRank.

Note that while this is quite similar to one of the topics of this thesis, the authors of this paper
only evaluate the social model on a small dataset of publications in the \ac{ACM} \ac{SIGIR} conference.
Because of this, the publications are tightly connected in topic and the social network is quite dense, unlike in our case.

\cite{aseo} describe inner workings of academic search engines on the example of Google Scholar and
discuss search engine optimization in the context of academic search engines. 
The reviewers of this paper mention the potentially harmful effect of \ac{ASEO} when the authors artificially increase 
the relevance of their publications to the search queries by deliberately overusing certain keywords.

\subsection*{Existing systems}
\addcontentsline{toc}{subsection}{Existing systems}

As the main goal of this thesis is improving the user experience of the Charles Explorer application,
we briefly mention the existing systems that are used for the academic research result retrieval and their features.

\textbf{Google Scholar} is a popular academic search engine maintained by Google LLC IPA.
According to the documentation\footnote{\url{https://scholar.google.com/intl/en/scholar/inclusion.html}}, the system crawls the Internet for well-formed academic publications and indexes them automatically. 
\cite{google-scholar-size-estimation-2014} estimated the size of the Google Scholar index to be around $99.8$ million documents in 2014.
Later estimates by \cite{google-scholar-size} put the size of the Google Scholar index at around $389$ million documents in 2019.

While the automated approach to the indexing of the documents allows for the inclusion of a large number of documents, 
\cite{predatory-google-scholar-scopus} claims it might lead to the inclusion of low-quality or predatory publications.

Because of the automatic indexing approach - and the possibility of irregular schema of the indexed documents stemming from this -
the Google Scholar index also does not feature search on metadata like author / organization affiliations.
Many publications are also missing the links to the author profiles (authors are often only mentioned by their name), 
which makes the exploration of the social network of researchers difficult.

\textbf{Scopus} is a subscription-based academic search engine maintained by Elsevier.
\cite{google-scholar-size-estimation-2014} estimated the size of the Scopus index to be around $53.4$ million documents in 2014.

According to the documentation\footnote{\url{https://www.elsevier.com/products/scopus/content/content-policy-and-selection}}, contributions to the Scopus index 
are actively curated by a team of field experts, which allows for the inclusion of high-quality publications.
Despite this effort, \cite{predatory-scopus} and others claim that the Scopus index still includes a significant number of predatory publications.

Because of the manual curation of the documents, the Scopus web application features a faceted search on various publication metadata
and relations between the authors, their organizations and the publications themselves.

\textbf{Web of Science} is a subscription-based academic search engine maintained by Clarivate Plc.
\cite{google-scholar-size-estimation-2014} estimated the size of the Scopus index to be around $56.9$ million documents in 2014.
Similarly to Scopus, indexing of new publications is actively curated by a team of field experts.

Thanks to the manual curation of the documents, the Web of Science web application also features a faceted search on publication metadata.

\cite{wos-scopus-not-global} compares the coverage of the Web of Science and Scopus indexes and
points out that neither of the indexes might not fairly represent the global scientific output.
According to the claims, the indices are biased towards the publications from the English-speaking countries,
written in English and concerning several specific fields of study.

\section*{Goals of the thesis}
\addcontentsline{toc}{section}{Goals of the thesis}

The main goal of this thesis is to explore the practical applications of social network analysis 
in the context of academic research. The overarching goal is to improve the usability
of the Charles Explorer application by using the social network analysis.

The first goal is to create a social network of researchers and their publications from the data available in the university's information system.
In this part, we want to devise an effective transformation of the relational data model into a graph data model, 
which will allow us to use the graph algorithms for the social network analysis.

The second goal is to explore the practical applications of the social network analysis in the context of academic research.
In this part, we evaluate the usability of the social network metrics for the re-ranking of the document retrieval results.

The final goal is to improve the visualization of the academic social network in the Charles Explorer application.
While the current implementation is sufficient for the basic exploration of the social network, 
it lacks the advanced features that would allow for the more detailed analysis of the social network.
The current state of the visualization also suffers from performance and UX issues, which we want to address in this thesis.

\section*{Experimental setup}

In various parts of this thesis, we are running different benchmarks and experiments.
While most of those do not measure the computational performance of the algorithms, we still want to 
provide the reader with the information about the hardware and software used in the experiments for full reproducibility.

If not stated otherwise, the experiments were run on a machine with the following specifications:
\begin{itemize}
    \item \textbf{APU}: AMD Ryzen 7 PRO 2700U
    \item \textbf{RAM}: 24 GB
    \item \textbf{OS}: Linux Mint 21 (5.15.0-112-generic)
\end{itemize}

Additionally, the following software versions were used in the experiments:

\begin{itemize}
    \item \textbf{Python} 3.10.12
    \item \textbf{SQLite} 3.45.1
    \item \textbf{Memgraph} v2.18.0
\end{itemize}

Where important (due to breaking changes across versions or the use of specific features), 
the versions of the libraries used in the experiments are listed in the \texttt{README} files of the respective repositories.

\section*{Blog links}
As the experiments evaluated in this thesis are quite extensive, 
the thesis itself contains only the most vital details about the experimental setup, outcomes and the conclusions for brevity.

For the full details about the experiments, the reader is encouraged to visit the blog posts that accompany this thesis.
These blog posts contain the full implementation details, the code snippets, the intermediate results of the experiments
or reasoning behind some of the less important design decisions that were made during the implementation.

In parts with relevant blog posts available, the links to the blog posts are provided with the following notation:\textsuperscript{\href{https://barjin.github.io/edu/thesis-blog/}{[blog]}}.

Note that the links to the blog posts are only available in the digital version of the thesis and the URLs are not visible in the printed version.
In case of reading the printed version of the thesis, the reader is encouraged to visit the blog at \url{https://jindrich.bar/edu/thesis-blog/}.