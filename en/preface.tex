\chapter*{Introduction}
\addcontentsline{toc}{chapter}{Introduction}

A \textbf{social network} is a theoretical construct describing relations between entities that are usually homogenous in nature.
This definition formed over years mostly from social sciences and psychology, where it was used to create an abstraction of the real-world social interactions between people.
In those fields, the social networks were used to study - among others - social interactions between pairs of people and their influence on the overall structure of the society.

Later on, the social network theory slowly permeated into other fields of study, including discrete mathematics and computer science.
The computing performance of modern computers and the theoretical advances in the field of graph theory allowed for the analysis of large datasets, which in turn allowed for the study of social networks on a larger scale.
This led to the creation of the field of \textit{social network analysis}.

While the theory behind social network analysis is already quite mature, its practical applications are still being explored.
The most prominent use cases include the analysis of social media platforms, where the social network is formed by the users of the platform and their interactions.
Unfortunately, perhaps to keep the competitive advantage, the social media platforms seldom share any details about their social network analysis methods.

Without the access to the user data of commercial social media platforms, we are left with the analysis of social networks that are publicly available.
One such example is the social network of academic researchers - individuals - and their publications - which represent their interactions between each other.
The realm of academic researchers and collaboration on different publications opens a multitude of opportunities for research - since both the ``nodes'' (the researchers) and the ``edges'' (publications) of the social network have interesting data attributes available for them. 

For the researchers, this can be e.g. affiliation with parts of the university, their academic title or their role within the university - e.g. are they lecturers, postdoc researchers, or graduate students helping out on one or two publications etc.

For the publications, the data can include the authors, the publication's affiliation with faculties, year of publishing, publication keywords or a ``topic'' - a categorical variable grouping multiple publications regarding the same topic. This can be inferred from the name and the abstract of the publication using NLP methods.

This thesis demonstrates practical use of social network theory in the context of academic research, a search for collaborators and topic discovery and explores the usability of the social network metrics for re-ranking of document retrieval results.

Lately, the term ``social network'' has been popularized as a synonym to the term \textbf{social media}.
These are online platforms that allow users to create a profile, share content and interact with other users.
Surprisingly enough, this overloading of the term is quite universal between languages - e.g. the Czech expression ``sociální síť'' and the German ``Soziales Netzwerk'' are used for both the social network theory and the social media platforms.
In the rest of this thesis, the term ``social network'' will be used to refer strictly to the theoretical concept or it's concrete representation in the form of a graph, while the term ``social media'' will be preferred for the online platforms.

\section*{Related work}
\addcontentsline{toc}{section}{Related work}

The social network analysis of research groups has been a topic of interest for various publications. 
\cite{ORDOOBADI2019S164} and \cite{CIMENLER2014667} explore the social network of researchers and their publications, and they try to infer the 
social network structure from the data about the co-authorship of the publications. 
Later on they try to compare the social network metrics with the academic performance - citation counts - of the researchers.

\cite{inproceedings} explore the use of social network data for improving the search result ranking in the context of information retrieval.
By using data from 

\section*{Goals of the thesis}
\addcontentsline{toc}{section}{Goals of the thesis}

The main goal of this thesis is to explore the practical applications of social network analysis 
in the context of academic research. The overarching goal is to improve the usability
of the Charles Explorer application by using the social network analysis.

The first goal is to create a social network of researchers and their publications from the data available in the university's information system.
In this part, we want to devise an effective transformation of the relational data model into a graph data model, 
which will allow us to use the graph algorithms for the social network analysis.

The second goal is to explore the practical applications of the social network analysis in the context of academic research.
In this part, we evaluate the usability of the social network metrics for the re-ranking of the document retrieval results.

The final goal is to improve the visualization of the academic social network in the Charles Explorer application.
While the current implementation is sufficient for the basic exploration of the social network, 
it lacks the advanced features that would allow for the more detailed analysis of the social network.
The current state of the visualization also suffers from performance and UX issues, which we want to address in this thesis.

