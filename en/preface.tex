\chapter*{Introduction}
\addcontentsline{toc}{chapter}{Introduction}

An introduction about the reasons for this research. 
While the theory behind social networks is already quite mature, they are not receiving widespread adoption yet.
The realm of academic researchers and collaboration on different publications opens a multitude of opportunities for research - since both the ``nodes'' (the researchers) and the ``edges'' (publications) of the social network have interesting data attributes available for them. 

\dots

For the researchers, this can be e.g. affiliation with parts of the university, their academic title or their role within the university - e.g. are they lecturers, postdoc researchers, or graduate students helping out on one or two publications etc.

For the publications, the data can include the authors, the publication's affiliation with faculties, year of publishing, publication keywords or a ``topic'' - a categorical variable grouping multiple publications regarding the same topic. This can be inferred from the name and the abstract of the publication using NLP methods.

\dots

This thesis demonstrates practical use of social network theory in the context of academic research, a search for collaborators and topic discovery and explores the usability of the social network metrics for re-ranking of document retrieval results.

\dots