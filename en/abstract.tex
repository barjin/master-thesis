%%% A template for a simple PDF/A file like a stand-alone abstract of the thesis.

\documentclass[12pt]{report}

\usepackage[a4paper, hmargin=1in, vmargin=1in]{geometry}
\usepackage[a-2u]{pdfx}
\usepackage[utf8]{inputenc}
\usepackage[T1]{fontenc}
\usepackage{lmodern}
\usepackage{textcomp}
\usepackage{amssymb}

\begin{document}

%% Do not forget to edit abstract.xmpdata.

The goal of this thesis is to explore the use of social networks in document retrieval in the context of academia. 
In the first part of the thesis, we infer various types of social networks from the freely accessible data about the publications and researchers affiliated with the Charles University. During this, we discover different metrics on the data, explaining their value for academic collaboration.
In the latter part of the thesis, we look into utilizing the social network model for re-ranking search results in the Charles Explorer web application. We also try to implement an experimental visualizing tool for the data as a part of the Charles Explorer application.
In the conclusion we address some issues we've encountered during the development and data exploration, and we suggest ideas for further research.

\parskip=1em

%% TODO

Tématem diplomové práce je analýza sociálních sítí ve volně přístupných datech Univerzity Karlovy. 

\checkmark \; \; Student při řešení navrhne efektivní způsob transformace relačních dat o entitách na UK na grafový model. 

Dále student navrhne postup pro analýzu různých aspektů vzniklé sociální sítě (např. hledání komunit, výpočet centrality vrcholů atd.) 

a jejich význam a specifika pro aplikaci v akademickém prostředí. 

Výsledky student následně porovná s existujícími algoritmy pro analýzu sociálních sítí. 

\checkmark \; \; Součástí práce je experimentální implementace nástroje pro vizualizaci a prohlížení vzniklé sociální sítě do aplikace Charles Explorer (https://explorer.cuni.cz). 

Student také zhodnotí využití sociální sítě pro zlepšení výkonu fulltextového vyhledávání (reranking výsledků) v aplikaci.

\end{document}
